\documentclass[twocolumn,a4paper,10pt]{article}
\usepackage{fancyhdr}
\usepackage{amsmath}
\usepackage{xcolor}
\usepackage{graphicx}
\usepackage{sectsty}
\sectionfont{\large}
\usepackage[left=1cm, right=1cm, top=2cm, bottom=2cm]{geometry}

\usepackage{listings}
\usepackage{latexsym}
\usepackage{proof}
\usepackage{amssymb}

%\lstset{
%  basicstyle=\footnotesize
%}

%% New commands
\newcommand{\eqdef}{\;\stackrel{\text{def}}{=}\;}

\begin{document}
\title{Research paper summary:\\
       ``De-indirection for Flash-based SSDs with Nameless Writes''\\
       \small{Y. Zhang, L. P. Arulraj, A. C. Arpaci-Dusseau, R. H. Arpaci-Dusseau}}
\author{Federico Wasserman \& Rodolphe Lepigre\\
        MOSIG - Parallel, Distributed and Embedded Systems}
\date{\today}
\maketitle

%\begin{abstract}
%This article summarizes the description and evaluation of a new device
%interface that reduce, if not remove, the need for indirection in flash-based
%SSD storage: It has been presented by Yiyang Zhang et al. a paper called
%\textit{De-Indirection for Flash-Based SSDs with Nameless Writes"}.
%
%Their idea is to let the device choose the most suitable block in which to
%write the data. The address of the write is then provided to the client (file
%system). By doing so, the device is able to execute critical tasks such as
%garbage collection and wear leveling without relying on the file system.
%
%We quickly present SSDs and the notion and indirection. We then introduce
%the Nameless Writes interface and some of the evaluations done by the authors
%of the original paper on an emulated nameless-writing device.
%
%This summary is in no way intended to cover the summarized paper in full.
%Readers interested in the details must refer to the original paper.
%\end{abstract}

\section*{Introduction}
Indirection is used in many computer systems. It is used mainly to improve
their performance and reliability. For example current hard drives make use of
a mapping table that remaps faulty sectors to other locations transparently
(from the file system's point of view).

Solid State Drives (SSDs) also use indirection in order to satisfy the
specific requirements of their hardware (wear leveling, erase / program
cycle, garbage collection). In particular, when a file is modified on an SSD,
its original location cannot be updated. The hardware uses copy on write and
must make use of indirection to make the filesystem believe that the updated
data still resides at the same (virtual) address. The problem is that storing
a table to map physical to virtual addresses wastes a lot of expensive flash
memory.

In this article, we summarize a paper called \textit{De-Indirection for
	Flash-Based SSDs with Nameless Writes}, in which Yiyang Zhang et al.
propose a way to modify to current SSDs and file systems to remove the need
for indirection. Their idea is to let the SSD choose the name (address) of the
data to be written and only after inform the file system for it to register
the location in its internal data structures.

\noindent
\textbf{Content of the report.} In the first section we introduce SSDs and their
particular functionning mode. We then present the techniques of indirection
that are currently used in SSDs to ensure wear-leveling in the second section.
In the third section, Nameless Writes are introduced together with their
device interface. The problem of recursive writes is also addressed.
Finally, we present some evaluation in the fourth section and show that
nameless writes could be a viable solution to solve the indirection issues
in SSDs.

\section*{SSD principles}
Although to the file system hard drives appear the same way independent of
their architecture, internally they can be quite different. SSDs do not have
the same structures and addressing modes as standard hard drives. They are
composed of several flash banks divided in blocks thus the normal
cylinder-head-sector addressing is not appropriate albeit an indirection is
done to hide this from upper layers.

Because of their inner components, a flash block must be erased before it can
be written. As such, data cannot be overwritten in place, so disks use a
pattern known as erase-program, requiring not overwritten pages to be moved
elsewhere while the block is being erased to be then put back with the new
ones.

A peculiarity of SSDs is that they have a limited number of writes to the same
block before starting to show high bit error rates. When this property is
mixed with the fact that data needs to be written using an erase-program
model, there is a writing multiplication that hurts deeply the SSD lifetime.

To counter these adversities, the firmware inside the SSD handles with
indirection where to write data and performs internal garbage collection to
recover the best possible block to rewrite to perform wear-leveling and
increase the disk lifetime. Although disk normally hide all their inner
working, after their introduction, file systems started to help them
including a command called TRIM. This command informs the disk which pages are
not in use anymore so as to aid in the garbage collection mechanism and do not
hurt the lifespan much.

\section*{Indirection in SSDs}
As discussed in the previous section, indirection plays a crucial role in
SSDs to increase their lifetime. Indirection is implemented in the Flash
Translation Layer (FTL), which maps the virtual to the physical address space. 

The main problem with the indirection found in FTLs is that it comes with
performance and space overheads.

To better understand the problem, let's consider that we have a 1TB SSD split
up into 2KB pages. A full page mapping (with one 32-bit pointer per page)
would require a 2GB space, which would be unacceptable considering the cost
of SRAM.

An other idea is to map blocks of, say, 128 pages. The space cost of the
mapping would be reduced to 32MB which seems more reasonable. However, it has
been shown that with such coarse grain mapping, there is a huge performance
overhead introduced by garbage collection (Gupta et al.).

A majority of the SSDs that are available to buy use both of these schemes.
Most of the data is mapped at block level, but there is a small area that is
mapped at the page level for more efficient updates. This technique keeps
the space overhead reasonably low while avoiding the high garbage collection
cost of the full block-level mapping. The main problem with this technique is
that the FTL is made very complex, and there is still a performance overhead
(and it is also harder to maintain). Moreover, garbage collection can still be
costly and induce a performance problem.

In all of these designs, the indirection in the FTL implies some cost that
might not be very significant now, but when SSDs scale, the hybrid schemes
will eventually become infeasible. To solve these problems, the authors of the
paper introduce Nameless Writes.

\section*{Nameless Writes}
The aim of the new technique presented here is to reduce most of the costs of
indirection while keeping its benefits. The idea is not to provide any address
for a write operation. The device can itself choose where to put the data
among all physical pages. The file system is then notified and can record the
data in its data-structures. Here is the main interface of a nameless writing
device:
\begin{lstlisting}
Nameless_Write(data, len): phys@
Nameless_Overwrite(phys@, data, len): new@
Physical_Read(phys@, len): data
Free(virt/phys@, len)
\end{lstlisting}
Note that the nameless overwrite has the same mechanism. The original address
is given as a parameter, and the new address is returned once the device has
made its choice. Also, since a nameless write is an allocating operation,
there is a corresponding free operation. In case of a nameless overwrite, the
free is handled by the device. Note that for the read operation the physical
address is to be provided.

There is but one problem with this Nameless Write interface: the recursive
update problem. If the only available write operation is the nameless one, any
update to the file system will require a recursive set of updates. For
exemple, appending a block at the end of a file will require the inode of the
file to be modified using a nameless overwrite which will change its location.
So any structure referencing the inode will have to be modified, propagating
potentially the update up to the file system's root.

To solve this problem, the authors of the paper propose to provide a segmented
address space: a large physical address space for nameless writes, and a small
virtual address space. Keeping the pointer-based structures in the virtual
space will break the recursion in case of reccursive update. The operations
provided for virtual read and write are usual:
\begin{lstlisting}
Virtual_Read(virt@, len): data
Virtual_Write(virt@, data, len)
\end{lstlisting}
Only a very small indirection table is kept by the device to map the virtual
addresses to physical addresses, and it will usually fit in the device cache.

Note that physical writes are not allowed, only the device has the power to
choose the location of a write. Also, to distinguish a block that is mapped by
a virtual block, a flag is set in the block headers.

The device also needs to be able to move data from one place to another in
order to allow wear leveling. The interface defines a callback function that
can notify the file system of data migration.
\begin{lstlisting}
Migration [Callback] (old_phys@, new@)
\end{lstlisting}
Blocks in the virtual address space can be moved without having to notice the
file system, but for the blocks that are in the physical space, the migration
callback have to be used for the file system to update its data structures
accordingly.

\section*{Evaluation}
To evaluate the device interface a series of simulators with different FTLs
were created. The first version uses a complete page-level mapping, the second
one a hybrid mapping and the third one the nameless-writing structure. The
devices try to emulate real life equipment in terms of performance, although
they perform somewhat better than today’s devices.

The first result presented is the fact that the FTL mapping table size is
smallest for Nameless writes when compared to the other two, as a result of
having only a small share of the disk that uses this structure. As for the
page-level mapping, it’s size is approximately 10\% of the total disk space
while the hybrid does not occupy a lot of space, still being bigger than
nameless.

In relation to the performance of the different disks, the results show that
nameless writing performs similar to page-level mapping, meaning close to the
upper-bound as in page-level we have a direct mapping to each page. Both
always show results surpassing the ones from a hybrid implementation,
specially when the writes are random and the hybrid cannot keep up with the
allocation cost.

Finally when considering wear-leveling techniques to reorder data and
migrating it to rewrite in unused pages, the results show that there is a
small overhead when using Nameless writes instead of pages. This result is
expected because of the model, but due to the small overhead and the frequency
that wear-leveling operations are done, they are not considered dominant and
negative.

\section*{Conclusion}
Nameless Writes have been introduced to reduce the cost of indirection in
SSDs. This new interface still ensures wear-leveling and has been tested using
and emulated device.

Nameless-writing devices have great advantages, as have been shows with the
results of the evaluation. In particular, the space cost of the system is
greatly reduced even compared to hybrid mapping FTLs. The design of the FTL is
also made much simpler.

Even though the authors only ported the Linux ext3 file system to work with
their emulated nameless-writing device, they argue that it would be possible
(and will be future work) to do the same with other file systems (ext2,
copy-on-write file systems, extent-based file systems).

\end{document}

